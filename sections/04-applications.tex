\section{\Tool{}-based applications}%
\label{sec:applications}

In this section, we build three applications on top of \tool{}, each a research
contribution in its own right.
%
First, we build an application that allows a service provider to disclose its
infrastructure configuration in a user-verifiable way, thus eliminating the
trust that users have to have in third-party infrastructure (\S~\ref{sec:vct}).
%
Second, we launch a Tor bridge inside an enclave, which mitigates several
classes of attacks that the Tor network has struggled with in the past
(\S~\ref{sec:tor-bridge}).
%
Third, we show that \tool{} can handle computationally expensive workloads by
moving a Web browser into an enclave and letting users interact with it via a
remote desktop environment (\S~\ref{sec:browser}).

\subsection{Verifiable configuration transparency}%
\label{sec:vct}

Service providers typically outsource their infrastructure to third-party
providers like content delivery networks or cloud computing vendors.  The way
these third-party providers are configured often affects user privacy.  For
example, a service provider may configure a third-party reverse proxy to strip
client IP addresses before requests are forwarded to the servers that are under
the service provider's control.  How can users know that the reverse proxy is in
fact configured to strip client IP addresses?  We built an enclave application
that solves this problem by disclosing infrastructure configuration in a
user-verifiable way.  The idea, illustrated in Figure~\ref{fig:vct} in
Appendix~\ref{sec:more-diagrams}, consists of a lightweight enclave application
whose sole purpose is to answer client requests by querying the API of the
third-party infrastructure provider.  We built our proof-of-concept
implementation for Cloudflare but the code is easily adapted for other
providers.

To interact with Cloudflare's API endpoint, one needs a confidential bearer
token for authentication and a semi-confidential zone ID~\cite{spectrum-config}.
Unlike the API endpoint's URL, these two values cannot be hard-coded in the
(public) source code of the enclave application.  We therefore add a second Web
server to the enclave application whose only purpose is to receive as input the
bearer token and the zone ID (cf.  \S~\ref{sec:secrets}).  We carefully
constrained this Web server's HTTP handlers, making it impossible to inject
anything into the enclave \emph{but} the bearer token and the zone ID.  We
further configured the proxy to only forward to this Web server connections
originating at the EC2 host.  Internet-connected adversaries cannot reach this
Web server.  After the administrator launches the enclave, she injects the
confidential values into the enclave by calling the private HTTP endpoint from
the EC2 host.  Users have no reason to be concerned about this secret endpoint
because the enclave application's source code shows that the secret values are
only used for the API request to Cloudflare.  Clients talk to this enclave
application via a single endpoint, which returns the JSON-encoded Cloudflare
configuration in the HTTP body, and the enclave's attestation document in a
custom HTTP header.  Section~\ref{sec:attestation} explained that the
attestation document must contain a client-provided nonce.  The client provides
this nonce as part of the HTTP request URL.  In summary, clients make the
following request:

\begin{lstlisting}[numbers=none,basicstyle=\small\ttfamily]
GET /verify?nonce=3a26d...a937f HTTP/2
Host: enclave.example.com
\end{lstlisting}

And the server responds with:

\begin{lstlisting}[numbers=none,basicstyle=\small\ttfamily]
HTTP/2 200 OK
Date: Mon, 13 Mar 2023 18:34:29 GMT
Content-Type: application/json
X-Attestation-Document: hEShATgioFkRJalpbW9kdWx...

{
  "domain": "service.provider.com",
  "modified_on": "2021-09-08T18:04:10.711156Z",
  ...
}
\end{lstlisting}

Upon receiving the enclave's response, the client first verifies the
authenticity of the attestation document (cf. \S~\ref{sec:attestation}).  Once
convinced that the enclave's response is authentic, the client inspects the
body of the response---which comes directly from Cloudflare's API.  In
particular, the client consults Cloudflare's API documentation to verify that
the service provider configured Cloudflare as promised.  Finally, the user
verifies that the domain inside the enclave's response matches the domain that
the service provider makes available to its users.

\subsection{Tamper-resistant Tor bridge}
\label{sec:tor-bridge}

The Tor network's security rests on the assumption that certain relays in a
user's circuit do not collude.  This assumption does not always hold, as in the
2014 attack that sought to deanonymize onion service
users~\cite{Dingledine2015a}.  The attack consisted of several malicious relays
that injected a sequence of Tor cells to encode a messages along the
circuit~\cite[\S~5.6]{tor-spec}.  Being an active attack, this required a
modified relay that deviated from the Tor protocol.  If done well, such attacks
can be difficult to spot.

\Tool{} can help mitigate such attacks by running Tor infrastructure inside an
enclave.  By taking advantage of remote attestation, Tor clients can rest
assured that they are communicating with an authentic Tor implementation that
does not deviate from the Tor protocol.  We demonstrate that this is possible
by setting up a Tor bridge inside an enclave.\footnote{We chose to set up a Tor
bridge instead of a relay because bridges can be configured to remain private
and therefore cause no harm to the network in case our implementation had
bugs.}

\textbf{Proof-of-concept deployment}: Running the Tor executable inside an
enclave is straightforward: the Dockerfile and startup script are nearly
identical to Listing~\ref{fig:example}.  Remote attestation however is more
complicated.  Ideally, Tor clients would attest the authenticity of their Tor
bridge as part of the Tor protocol itself but for the sake of this prototype, we
are content with handling remote attestation outside the Tor protocol: Once the
Tor bridge is done bootstrapping, it registers its long-term identity key with
\tool{}.  The enclave then exposes two TCP ports: port 443 for \tool{} and port
9001 for Tor.  A Tor client first remotely attests the enclave, followed by
establishing a circuit over the bridge and---while establishing the
circuit---verifies that the bridge's long-term identity key is identical to the
key in the attestation document.  If so, the client can rest assured that it's
talking to a publicly verifiable Tor bridge.

We implemented the aforementioned prototype and configured Tor Browser v12.0.1
to use our in-enclave bridge.  Using this setup, we were able to watch 4K
YouTube videos without buffering delays.
%
The above setup works well for an ad-hoc setup but is insufficient for
network-wide deployment of in-enclave relays and bridges.  In this case, relays
and bridges need a way to announce if they support relay attestation.  This is
the job of the Tor network's consensus, which is generated every hour by the
distributed directory authorities.  Changes to the network consensus are
complex and need to be addressed in the protocol specification and Tor's
reference implementation.

\textbf{Comparison to SGX-Tor}: In their NSDI'17 paper, Kim et al. augmented the
Tor code with Intel SGX, thus giving clients, relays, and directory authorities
the ability to remotely verify each other~\cite{Kim2017a}.  Our approach differs
in the following aspects:
%
Clients that seek to verify an SGX enclave's attestation document need to talk
to Intel's attestation service, which brings with it an array of privacy
problems~\cite[\S~1.2]{Chen2019a}.  With Nitro Enclaves, clients can verify an
attestation document without having to contact a third party, provided that they
have a copy of Amazon's root CA public key.
% End-to-end correlation attacks.
Next, the authors envision the entire Tor network to take advantage of SGX,
which is not feasible in our approach: Nitro Enclaves can only run in
Amazon-controlled AWS.  If all Tor relays ran inside AWS, Amazon would see both
traffic entering and exiting the network---ideal conditions for end-to-end
correlation attacks.  We therefore believe that only select Tor bridges benefit
from running in \tool{}, lest anonymity is jeopardized.
% Complexity of making it work.
As for practicality, Kim et al. had to go to great lengths to patch Tor to
support SGX~\cite[\S~5]{Kim2017a}.  Our proof-of-concept implementation took one
afternoon.  Finally, since Kim et al.'s paper was published, Intel announced the
discontinuation of SGX support for consumer-grade Core CPUs, which further
limits the number of SGX-capable Tor clients.

\subsection{An in-enclave Web browser}%
\label{sec:browser}

The previous enclave application focused on a low-latency, high-throughput use
case.  We now build an application that requires low latency in addition to
being \emph{computationally demanding}: we move a modern Web browser into an
enclave, thus isolating the browser from its user's desktop environment.
% Advantages
If a browser is compromised by, say, a malicious Web site, then the malicious
code is constrained to the enclave, unable to interact with the user's desktop
environment.  In addition, users benefit from not having to provide the
ever-increasing computational resources to run the browser.
% Disadvantages
The downside of this approach is that the enclave's EC2 host gets to see the
browser's traffic---but this can be solved by using a VPN or Tor.
Alternatively, the pervasive deployment of HTTPS helps protect page contents
from the EC2 host's prying eyes and one could configure the browser to use
DNS-over-HTTPS to further protect DNS traffic.

% How we built this
We start with an Ubuntu Docker image, which we extend by installing basic X11
graphical utilities, OpenSSH, TigerVNC, the i3 window manager, and a Chromium
browser, all via Ubuntu's package manager.\footnote{We used Ubuntu 20.04.4 from
docker.io.}  We used OpenSSH port forwarding to tunnel TigerVNC's TCP traffic,
which prevents the EC2 host from spying on VNC traffic.  The enclave's SSH
public key acts as the root of trust, and it lives and dies inside the enclave.
We configure \tool{} to add a hash over this public key to the attestation
document.  Before establishing an SSH connection, clients fetch the enclave's
attestation document from the \tool{} port to learn what SSH public key to
expect.
%
By default, AWS's tooling for Nitro Enclaves assumes applications smaller than
what we need for a remote desktop environment.  Depending on the instance size
and tool version, we sometimes had out-of-memory errors converting container
images to enclave images.  Additionally, the enclave manager allows a maximum
allocation of 512 MB to each container, which isn't enough to run most graphical
desktop software by default.  This limit needs to be raised.

There is plenty of room for improvement, both in reducing the enclave's image
size and in reducing the latency that our VNC server induces.  One can also take
advantage of AWS's numerous data centers by launching the browser enclave in a
region that is close to the user, to further minimize latency.

\textbf{Subjective user experience}:
Upon using a VNC client to interact with the in-enclave Web browser, we found
that navigation was relatively painless but large screen updates were spread
over a noticeable amount of time, usually obscuring the underlying animation.
Video played at less than full framerate, and VNC doesn't support audio
playback.  However, the user-perceptible latency is dominated by the round-trip
time between the client and the enclave.  In any case, this application
demonstrates that the enclave itself is no barrier to achieving similar
performance to any other remote desktop service.

\textbf{Alternative approaches}:
Wang et al.'s WebEnclave work sets out to protect Web sites from malicious
browser extensions: Web developers can use the \texttt{<web-enclave>} tag to
instruct the browser to execute code inside WebEnclave~\cite{Wang2021a}.  Unlike
WebEnclave, our application protects the user's operating system from the
browser, instead of Web pages from browser extensions.
