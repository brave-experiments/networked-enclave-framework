\section{A basic example}%
\label{sec:example}

Figure~\ref{fig:example} illustrates how an enclave application (called
\texttt{enclave-app}) can run alongside \tool{}.  Figure~\ref{fig:dockerfile}
shows a Dockerfile that adds \tool{}, the enclave application, and a start
script to the image, followed by launching the start script, which is
illustrated in Figure~\ref{fig:start}.  All the script does is first launch
\tool{} in the background followed by launching the enclave application.  If
the application builds reproducibly, it is possible to run it inside an enclave
\emph{without modifications}.

\begin{figure}[t]
  \begin{subfigure}[b]{\linewidth}
    \centering
    \begin{lstlisting}
    FROM alpine:latest

    COPY nitriding /bin/
    COPY enclave-app /bin/
    COPY start.sh /bin/

    CMD ["start.sh"]\end{lstlisting}
    \caption{A Dockerfile that embeds \tool{} along with the enclave
      application, \texttt{enclave-app}.}
    \label{fig:dockerfile}
  \end{subfigure}

  \begin{subfigure}[b]{\linewidth}
    \centering
    \begin{lstlisting}[language=bash]
    #!/bin/sh

    # Launch nitriding in the background.
    nitriding \
      -fqdn "example.com" \
      -acme \
      -appwebsrv "http://127.0.0.1:8080" &

    # Launch the application.
    enclave-app\end{lstlisting}
    \caption{The start.sh shell script launches \tool{} in the background,
    followed by launching the enclave application}
    \label{fig:start}
  \end{subfigure}

  \caption{An example of how a simple enclave application can be bundled with
  \tool.}
  \label{fig:example}
\end{figure}

\section{Architectural diagrams}%
\label{sec:more-diagrams}

Figure~\ref{fig:vct} illustrates the design of our enclave application which
implements verifiable configuration transparency.

\begin{figure}
  \centering
  \begin{tikzpicture}[node distance=20pt] % Reduce default distance to save space.

  \node [draw,
         label={[anchor=north]above:EC2 host},
         minimum height=90pt,
         align=center,
         minimum width=60pt] (ec2) {};

  \node [draw,
         label={[anchor=north]above:Enclave},
         right=0pt of ec2,
         fill=black!10,
         minimum height=90pt,
         minimum width=60pt] (enclave) {};

  \node [draw,
         below=0pt of enclave.south west,
         minimum width=120pt] (hypervisor) {Hypervisor};

  \node [draw,
         align=center,
         below=of ec2.north] (proxy) {Proxy};

  \node [draw,
         align=center,
         below=of proxy.south] (admin) {Admin};

  \node[draw,
        align=center,
        fill=white,
        xshift=5pt,
        right=of proxy] (nitriding) {Nitriding};

  \node[draw,
        align=center,
        fill=white,
        below=of nitriding] (app) {App};

  \node [draw,
         left=of proxy] (cloudflare) {Cloudflare};

  \node [draw,
         below=of cloudflare] (client) {Client};

  % The admin configures confidential information.
  \draw [-latex]
        (admin.north) -- (proxy.south)
        node [midway, fill=white, circle, inner sep=-2pt] {\ding{202}};
  \draw [-latex]
        ([yshift=-5pt]proxy.east) -- ([yshift=-5pt]app.west);

  % Clients talking to the application.
  \draw [-latex, densely dotted]
        (client.east) -- ([yshift=-3pt]proxy.west)
        node [midway, fill=white, circle, inner sep=-2pt] {\ding{203}};
  \draw [-latex, densely dotted]
        ([yshift=5pt]proxy.east) -- ([yshift=5pt]nitriding.west);
  \draw [-latex, densely dotted]
        (nitriding.south) -- (app.north);

  % Application talking to Cloudflare.
  \draw [-latex, densely dashed]
        (app.west) -- (proxy.east)
        node [midway, fill=white, circle, inner sep=-2pt] {\ding{204}};
  \draw [-latex, densely dashed]
        ([yshift=3pt]proxy.west) -- ([yshift=3pt]cloudflare.east);

  % Application asks for an attestation document.
  \draw [-latex, densely dashdotted]
        (app.south) -- ([xshift=32pt]hypervisor.north)
        node [midway, fill=white, circle, inner sep=-2pt] {\ding{205}};

\end{tikzpicture}

  \caption{An overview of the enclave application that provides verfiable
  configuration transparency.  After launching the enclave, the operator
  configures the confidential bearer token and zone ID~(\ding{202}).  Clients
  can then request the service provider's Cloudflare
  configuration~(\ding{203}).  The application makes an HTTP request (containing
  the bearer token and zone ID) to Cloudflare's API~(\ding{204}).  Finally, the
  application asks its hypervisor for an attestation document~(\ding{205}) and
  embeds the attestation document in the response to the client, along with
  Cloudflare's response.}%
  \label{fig:vct}
\end{figure}
