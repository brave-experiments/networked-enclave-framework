\section{Related Work}
\label{sec:related-work}

Arnautov et al. present in their OSDI'16 paper a mechanism that allows Docker
containers to run in an SGX enclave~\cite{Arnautov2016a}---conceptually similar
to Nitro enclaves, which are effectively compiled Docker images.
%
In their 2022 arXiv report, King and Wang~\cite{King2022a} propose HTTPA---an
SGX-based extension to HTTP that makes a Web server attestable to clients.  Our
framework also allows for attestable Web services, but without modifications to
HTTP.

\paragraph{Applications of Enclaves}

Researchers have proposed numerous and diverse enclave-enabled systems, ranging
from DeFi oracles~\cite{Zhang16a}, to health apps for
COVID-19~\cite{Mailthody21a}, to networking middleboxes~\cite{Han17a}.  Despite
avid interest in academia, large-scale, real-world deployments of enclaves are
sparse.  In 2017, the Signal secure messenger published a blog post on private
contact discovery~\cite{Marlinspike17a}, which makes it possible for Alice to
discover which of the contacts in her address book use Signal without revealing
her contact list.  The Signal team accomplished this by relying on an SGX
enclave that runs the contact discovery code.  Two years later, in 2019, the
Signal team built its ``secure value recovery'' feature on SGX as
well~\cite{Lund19a}.

\paragraph{Frameworks for Enclave Development}

To facilitate working with enclaves, several frameworks have emerged that
abstract away complicated and error-prone low-level aspects of enclaves.
Examples are Asylo~\cite{asylo} and Open Enclave~\cite{openenclave}---both
libraries are implemented in C/C++ and are hardware agnostic, meaning that the
``enclave backend'' can be switched from, say, TrustZone to SGX.  While
frameworks render enclave development more convenient, memory unsafe languages
like C and C++ make it dangerously easy to introduce memory corruption bugs
that jeopardize the security of the enclave~\cite{Lee2017a}.  Cognizant of
this issue, Wang et al. implemented a performant Rust layer on top of Intel's
C++-based SGX SDK, making it possible to develop memory-safe applications in
SGX~\cite{Wang2019a}.

Our framework is built in the memory-safe Go programming language, which
eliminates an entire class of bugs that could jeopardize the security of
enclave applications, and unlike Asylo and Open Enclave, our framework only
supports Nitro enclaves because the security guarantees of a framework are only
as strong as the underlying enclave hardware, and in the case of Intel, ARM,
and AMD, side channel attacks are a severe concern.

\paragraph{Attacks Against Enclaves}

Enclaves based on Intel's SGX technology share a CPU with untrusted code, which
raises the flood gates for side channel attacks.  Consequently, attacks have
taken advantage of
speculative execution~\cite{VanBulck2018a,VanSchaik2021a},
branch ``shadowing''~\cite{Lee2017b},
the interface between SGX and non-SGX code~\cite{Bulck19a},
software faults~\cite{Murdock2020a},
shared caches~\cite{Brasser2017a},
and memory management~\cite{Wang2017a}.
Despite the considerable number of practical attacks, there is opportunity to
strengthen SGX against side channel attacks.  Oleksenko et al. introduce in
their ATC'18 paper a system that protects unmodified SGX applications from side
channel attacks by executing the enclave code on a CPU separate from the
untrusted code.  Note that this is the default for Nitro enclaves.

For a comprehensive overview of attacks against SGX, refer to Fei et al.'s
survey~\cite{Fei2021a} and Nilsson et al.'s arXiv report~\cite{Nilsson20a}.

Among all currently-available commodity enclaves, Intel's SGX has received the
most attention from academia but ARM's TrustZone and AMD's SEV have not been
spared and share SGX's conceptual security flaws.  In a CCS'19 paper, Ryan
demonstrates an attack that exfiltrates ECDSA private keys from Qualcomm's
implementation of a hardware-backed keystore which is based on
TrustZone~\cite{Ryan2019a}.  Similarly, Li et al. showed in a USENIX
Security'21 paper how an attacker can exfiltrate private keys from AMD
SEV-protected memory regions.  In a CCS'21 paper, Li et al. showed how an
attacker-controlled VM can read encrypted page tables, and how an attacker can
create an oracle for encryption and decryption.

While Nitro enclaves are still young and have received nowhere near the same
scrutiny as SGX and friends, we believe that their dedicated hardware resources
provides stronger protection from side channel attacks than enclaves
that are based on shared CPU resources.
