\section{Introduction}

% What's the problem that we're trying to solve.
First introduced in 2015, Intel's Software Guard Extensions (SGX) technology
inspired diverse applications but also increasingly sophisticated attacks:
researchers successfully adapted speculative execution
attacks~\cite{VanBulck2018a}, injected software faults~\cite{Murdock2020a}, and
exploited side channels introduced by shared caches~\cite{Brasser2017a}, all
with the goal of exfiltrating information that was meant to remain in the
enclave.  The underlying flaw that most attacks take advantage of is that the
untrustworthy operating system and the enclave share a CPU, which provides many
options for side channel attacks.

% How Nitro enclaves are better.
In 2020, several cloud providers began offering ``confidential computing''
solutions; Google's is based on AMD's Secure Encrypted Virtualization
(SEV)~\cite{googlecc} while Microsoft's is based on SGX~\cite{azurecc}.  Both
offerings inherit the attack classes that plague their respective architectures.
Amazon took a different path by offering a new enclave architecture
based on their Nitro virtual machine isolation technology~\cite{nitro-enclaves}.
Nitro enclaves are separate virtual machines with hardware-enforced CPU,
memory, and device isolation, which imposes limits on access by untrustworthy code.
While the architecture appears promising, Nitro enclaves
remain difficult to use: documentation is sparse, few applications exist, and
enclaves can only interact with the parent EC2 instance via a constrained,
socket-like interface.  This paper presents the design, implementation, and
real-world application of a software framework that facilitates the development
of networked Nitro enclaves.  Key features of our framework include
(\emph{i}) the ability for enclave code to seamlessly and safely access the
Internet;
(\emph{ii}) a design for the horizontal scaling of enclaves by synchronizing
secret key material; and
(\emph{iii}) a reproducible build system and tooling that allows users to
remotely verify an enclave's authenticity.

% Challenges that we had to overcome.
During the development of our framework, we had to overcome several challenges.
First, Nitro enclaves are meant to be highly constrained environments and
therefore lack a dedicated networking interface.  We designed a mechanism that
allows enclaves to seamlessly send and receive data over the Internet while
maintaining an allow list of destinations, for defense in depth in case of
enclave compromise.
%
Second, the attestation process for Nitro enclaves was not designed to be
performed over the Internet.  We developed a way to bind a TLS session to
an attestation document to assure clients that they are communicating with
an authentic enclave.
%
Third, we had to devise a reproducible and yet easy-to-use build pipeline that
allows end users---regardless of their operating system---to compile the enclave
application and end up with the exact same image ID as the enclave provider.
%
Fourth, there is no out-of-the-box way for enclaves to scale horizontally while
synchronizing heir key material.  We therefore designed a mechanism that allows
enclaves to securely share their key material.

% Evaluation.
Having overcome the above design challenges, we implemented an easy-to-use Go
framework that abstracts away the difficulties and pitfalls of working with
networked enclaves.  The use of Go allows for rapid prototyping and greatly
reduces the risk of memory corruption bugs because of Go's memory safety.  We
conduct latency measurements to show that our framework can handle
high-throughput and real-time applications, and we demonstrate its usefulness
and robustness by building two applications on top of it.

\paragraph{Contributions}

This work makes three core contributions.

\begin{itemize}
  \item The design and implementation of a freely available Go framework that
    facilitates the implementation and deployment of enclave applications.  The
    framework consists of a library that an application can use to run as an
    enclave, and tooling that facilitates deterministic builds and seamless
    communication with the secure enclave.

  \item We make it possible via our framework to turn enclaves into networked
    applications that can easily scale horizontally to respond to increases in
    load.

  \item We build two real-world applications on top of our framework, the first
    of which---a remotely verifiable IP address pseudonymization system---is a
    contribution in its own right.
\end{itemize}

\paragraph{Structure}

Section~\ref{sec:background} provides background on secure enclaves in general
and AWS Nitro enclaves in particular.  Section~\ref{sec:design} introduces the
design and implementation of our software framework in addition to the build
process that guarantees reproducible enclave application builds, followed by
Section~\ref{sec:applications}, which presents two production-quality
applications that we built on top of our framework.  We evaluate our framework
in Section~\ref{sec:evaluation} and discuss its limitations in
Section~\ref{sec:limitations}.  Finally, Section~\ref{sec:related-work}
contrasts our work with past research.
