\section{Introduction}

% What's the problem that we're trying to solve.
First introduced in 2015, Intel's SGX technology inspired both numerous use
cases and increasingly sophisticated attacks.  Researchers successfully adapted
speculative execution attacks~\cite{VanBulck2018a}, injected software
faults~\cite{Murdock2020a}.
% TODO: Add more examples.
The core problem that most attacks take advantage of is that the untrustworthy
operating system and the secure enclave \emph{share a CPU}, which opens the
flood gates for side channel attacks.

% How Nitro Enclaves are better.
Google's and Microsoft's offerings are built on top of SGX and TrustZone, and
therefore inherit side channel attacks that plague the respective architecture.
Of particular interest is AWS's Nitro Enclaves because enclaves are assigned
dedicated hardware, and don't share a CPU with the untrustworthy operating
system.  Despite the more secure design, Nitro Enclaves are difficult to use.
Documentation is sparse, few applications exist so far, and enclaves can only
interact with the outside world via a constrained VSOCK interface.

% Our solution to the aforementioned problems.
In this work, we present the design, implementation, and real-world application
of a software framework that facilitates the development of networked secure
enclaves, i.e., applications that run inside a secure enclave while being able
to talk to endpoints on the Internet.  Among other features, our framework
(\emph{i}) makes it possible for clients to remotely attest the enclave's
authenticity; (\emph{ii}) allows for horizontal scaling of enclaves by
synchronizing secret key material; (\emph{iii}) and abstracts away the
constrained VSOCK interface between host and enclave.  Our framework builds on
top of the recently-introduced AWS Nitro Secure Enclaves~\cite{nitro-enclaves}.
Unlike Intel's SGX technology, Nitro Enclaves have dedicated CPUs that are not
shared with the host operating system, thus eliminating side-channel attacks,
which have plagued SGX for a long time~\cite[\S~III]{Nilsson20a}.  While some
aspects of our framework are specific to AWS, the protocol designs generalize
and could be adapted for other types of enclaves.

% Challenges that we had to overcome.
During the development of our framework, we had to overcome several challenges.
First, AWS Nitro Enclaves are meant to be highly constrained environments and
therefore lack a dedicated networking interface.  We equip enclaves with the
ability to receive and establish networking connections while maintaining an
allow list of destinations, for defense in depth in case of compromise.
%
Second, the attestation process was not designed to be done over the Internet.
To assure clients that they are communicating with an authentic enclave, we had
to find a mechanism to bind a TLS session to an attestation document.
%
Third, we had to devise a reproducible and yet easy-to-execute build pipeline
that allows clients---regardless of their operating system---to compile the
application that is running inside the enclave and end up with the same checksum
that the service provider ends up with.
%
Fourth, there is no out-of-the-box way for enclaves to scale horizontally while
synchronizing their key material.  We therefore designed a mechanism that allows
enclaves to synchronize their key material.

% Evaluation.
Having overcome all these challenges, we implemented an easy-to-use Go framework
that abstracts away the nuances of working with networked enclaves.  

No memory corruption because of Go



We show our
framework's potential by conducting performance measurements, and we demonstrate
its use and robustness by building three production-quality applications on top
of it: (\emph{i}) an IP address anonymization service, (\emph{ii}) a
$k$-anonymity service that is part of a privacy-preserving telemetry system, and
(\emph{iii}) a service that provides epoch-based randomness for clients.  We
deployed our first application, the IP address anonymization service, to TODO
users and report on our deployment and operational experience.

\paragraph{Contributions}

This work makes three core contributions.
%
First, the design and implementation of a freely available Go framework that
facilitates the implementation and deployment of enclave applications.  The
framework consists of a library that an application can use to run as an
enclave, and tooling that facilitates deterministic builds and seamless
communication with the secure enclave.
%
Second, we make it possible via our framework to turn enclaves into networked
applications.
%
Third, we demonstrate the use of our framework by applying it in three
production-quality code bases; (\emph{i}) in a system that anonymizes client IP
addresses, (\emph{ii}) to run an OPRF service, and to (\emph{iii}) run a server
that's part of a private telemetry system.  We further evaluate our prototypes
with respect to performance and security---especially related to code
complexity.

\paragraph{Structure}

Section~\ref{sec:background} provides background on secure enclaves in general
and AWS Nitro Enclaves in particular.  Section~\ref{sec:design} introduces the
design and implementation of our software framework in addition to the build
process that guarantees reproducible enclave application builds, followed by
Section~\ref{sec:applications} which presents three production-quality
applications that are built on top of our framework.  We measure our framework's
networking performance in Section~\ref{sec:evaluation} and discuss its
limitations in Section~\ref{sec:discussion}.  We conclude our work in
Section~\ref{sec:related-work} by putting it in the context of past work.
