\section{Introduction}

% What's the problem that we're trying to solve.
First introduced in 2015, Intel's Software Guard Extensions (SGX) technology
inspired diverse applications but also increasingly sophisticated attacks:
researchers successfully adapted speculative execution
attacks~\cite{VanBulck2018a}, injected software faults~\cite{Murdock2020a}, and
exploited side channels introduced by shared caches~\cite{Brasser2017a}, all
with the goal of exfiltrating information that was meant to remain in the
enclave.  The underlying flaw that most attacks take advantage of is that the
untrustworthy operating system and the enclave share a CPU, which invites side
channel attacks.  On top of that, SGX remains difficult to use and imposes
significant resource constraints~\cite{Ngoc2019a}.

% How Nitro enclaves are better.
Several cloud providers began in 2020 to offer ``confidential computing''
solutions.  Google's is based on AMD's Secure Encrypted Virtualization
(SEV)~\cite{googlecc} while Microsoft's is based on SGX~\cite{azurecc}.  Both
offerings inherit the attack classes that plague their respective architectures.
Amazon took a different path by offering a new enclave architecture based on
their custom Nitro system~\cite{nitro-enclaves}.  Nitro enclaves are
virtual machines with hardware-enforced CPU, memory, and device isolation.
%
While the architecture appears promising, Nitro enclaves are not meant to run
networked services and they remain difficult to use because enclaves can only
interact with the parent EC2 instance via a constrained, socket-like interface.
% Research questions.
This paper's contribution lies in answering the following research questions.
\begin{description}

  \item[$RQ_1$] How can we run arbitrary applications in Nitro enclaves without
    sacrificing security and without modifying the application?

  \item[$RQ_2$] What workloads (in latency, throughput, and computation) can
    Nitro enclaves sustain?

  \item[$RQ_3$] What new enclave application types emerge?

\end{description}

% How we answer RQ1.
We answer $RQ_1$ by designing and implementing \tool{}: a tool kit for rapidly
developing networked Nitro enclave applications.  \Tool{} abstracts away the
complexity and pitfalls of working with Nitro enclaves, making it possible to
run unmodified Docker images inside an enclave.  We equip \tool{} with a secure
networking channel between clients and the enclave and we further push the
envelope by building a user-verifiable enclave synchronization mechanism that
enables the scaling of enclaves, to handle intense, parallel work loads.

% How we answer RQ2.
Having developed \tool{}, we move on to answering $RQ_2$.  We study the
latency and throughput guarantees of Nitro enclaves in general and \tool{} in
particular.  We find that Nitro enclaves impose overhead in both latency
($\approx$1.2x) and throughput ($\approx$7x) compared to a system running in
Docker but this overhead still allows for low-latency and high-throughput
enclave applications.

% How we answer RQ3.
We answer $RQ_3$ by putting \tool{} to the test.  We built three novel enclave
applications that take advantage of \tool{}'s flexiblity and performance.  The
first application is a Tor bridge running inside an enclave, which allows users
to verify (using remote attestation) that the bridge is running
authentic and unmodified Tor code.  We found that this enclave-enabled Tor
bridge allows for smooth Web browsing---2160p YouTube videos played smoothly
and without buffering.  The second application allows a service provider to
grant its users insight into previously-confidential infrastructure
configuration.  By deploying a ``configuration query service'' inside an
enclave, users have proof that the service provider's privacy promises are
reflected in infrastructure configuration.  Finally, we launch a Web browser
inside an enclave, which isolates the user's main desktop environment from the
browser.
