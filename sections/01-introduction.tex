\section{Introduction}

% Problem statement.
Real-world enclave deployments tend to fall short in either providing security
(e.g., resisting side channel attacks), flexibility (i.e., supporting diverse
use cases and being easy to develop), or powerfulness (i.e., enabling
computationally demanding applications).  While reasonably flexible and
powerful, Intel's SGX failed to be secure because of its susceptibility to side
channel attacks~\cite{VanBulck2018a,Murdock2020a,Brasser2017a}.  Newer
developments like Apple's Secure Enclave appear secure but is limited to cell
phones and therefore lacks powerfulness and flexibility.

% Our solution to the problem.
In this work, we propose \tool{}, a free software tool kit that satisfies the
aforementioned properties.  By building on top of the recently proposed Nitro
Enclaves~\cite{nitro-enclaves}, \tool{} inherits its strong security properties.
Unlike Intel's SGX, Nitro Enclaves run on dedicated CPU and memory that is not
shared with untrusted code, which promises to mitigate side channel attacks.
%
By default, Nitro Enclaves are however severely constrained and difficult to
develop applications for because (i)~Nitro Enclaves are not meant to run
networked applications; (ii)~remote attestation is not designed to involve third
parties; (iii)~and enclave features like horizontal scaling require proprietary
Amazon technology whose privacy promises cannot be verified.  Our work abstracts
away these constraints, to make \tool{} flexible and user-verifiable.  Our tool
kit provides enclave applications with seamless networking via a tunneled
network interface, and it facilitates the creation of secure and authenticated
channels based on HTTPS.  We also design a new mechanism for the horizontal
scaling of enclaves, all while empowering third parties (i.e., users) to audit
these features using remote attestation.

% Enclave applications.
To demonstrate \tool{}'s usefulness, we put it to the test by developing three
new enclave application, each a research contribution in its own right.
%
First, we build an application that allows a service provider to ``publish'' its
infrastructure configuration, allowing the provider's users to verify that
privacy promises (e.g., the stripping of IP addresses) are properly configured.
%
Second, we show how a Tor bridge\footnote{A Tor bridge is a ``private'' Tor
relay whose purpose is to help users circumvent Internet censorship.}
can be run inside an enclave, which protects its users from protocol-level
attacks such as the infamous 2015 attack run by CMU~\cite{Dingledine2015a}.
We found that this enclave-enabled Tor bridge allows for convenient Web
browsing---4K YouTube videos played smoothly and without buffering.
%
Finally, to show that \tool{} can sustain computationally demanding low-latency
use cases, we launch a Chromium browser inside an enclave and make it available
to users via a remote desktop interface.  This protects users from browser
exploits by constraining the browser to a virtual machine that's physically
separate from the user's machine.

% Contributions.
Summing up, this work makes the following contributions:

\begin{itemize}

  \item We design and implement \emph{\tool{}}, a free software enclave tool kit
    that enables the rapid development of secure and computationally demanding
    enclave applications.

  \item We evaluate the latency and throughput guarantees of both \tool{} and
    the underlying Nitro Enclaves, finding that low-latency and high-throughput
    applications are possible despite there being room for much improvement.

  \item We put \tool{} to the test by developing three new enclave applications,
    finding that these applications can comfortably support computationally
    demanding, low-latency, and high-throughput use cases, all while enabling
    rapid prototyping.

\end{itemize}

Next, we provide background (\S~\ref{sec:background}), we present \tool{}'s design
(\S~\ref{sec:design}), we propose three enclave applications built on top of
\tool{} (\S~\ref{sec:applications}), we evaluate \tool{}'s latency and
throughput (\S~\ref{sec:evaluation}), we discuss its limitations
(\S~\ref{sec:limitations}), we discuss related work (\S~\ref{sec:related-work}),
and we finally conclude this paper (\S~\ref{sec:conclusion}).
