\section{Limitations}%
\label{sec:limitations}

We conclude the discussion of our framework by summarizing its limitations.

% Everybody needs to trust Amazon.
An obvious limitation is the reliance of our framework on Amazon, which acts as
the root of trust.  We mentioned in Section~\ref{sec:assumptions-overview} that
all parties must trust Amazon.  Note that this is not a new limitation of secure
enclaves---SGX-based applications must trust Intel while TrustZone-based
applications must trust ARM.  Despite the lack of alternatives, placing one's
trust in a single corporation's proprietary technology is problematic.

% Laypeople cannot audit enclave code.
Our system fundamentally relies on at least some users auditing the service
provider's application that runs in a secure enclave.  Needless to say, not all
users have the skills to audit the service provider's application and convince
themselves that the code is sound.  In fact, even among the subset of users that
are programmers, only a fraction may feel comfortable auditing source code for
vulnerabilities.  So what are the non-programmers to do?  We envision users to
congregate in forums where matters related to the service providers are
discussed.  A tech-savvy subset of the users is going to organize code reviews
and make public their findings.  Non-technical users may then trust other users
that audited the source code, but this is no different from most other software:
nobody audits all the software that they use, ranging from the kernel to the
myriad of user space applications, even when source code is available.

% One can hide bugs in plain sight.
The Underhanded C Coding Contest~\cite{underhanded-c} was about implementing
benign-looking code that was secretly malicious.  The contest attracted numerous
impressive submissions which showed that it can be surprisingly difficult to
find bugs \emph{even if one knows} that there is a bug in a given piece of code.
Analogously, the service provider could try to hide subtle, yet critical bugs in
the code to exfiltrate information from the enclave.  On top of that, if the
service provider ever gets caught, it may have plausible deniability and pretend
that the exfiltration bug was an honest programming error.  While we are unable
to solve this class of attacks, we can mitigate it by keeping the trusted
computing base in the enclave as small as possible.
