\section{Limitations}%
\label{sec:limitations}

% Everybody must trust Amazon.
An obvious limitation of \tool{} is its reliance on Amazon, which acts as the
root of trust.  Our trust assumptions state that all parties must trust Amazon.
Placing one's trust in a single corporation's proprietary technology is
problematic but this is a common limitation of enclaves---SGX-based applications
must trust Intel while TrustZone-based applications must trust ARM.

% Laypeople cannot audit enclave code.
Our system relies on at least some users auditing the service provider's enclave
application.  Needless to say, not many users have the skills to audit code.  In
fact, even among programmers, only a fraction may be qualified to audit source
code for vulnerabilities.  So what are the non-programmers to do?  We envision
users to congregate in forums where matters related to the service provider are
discussed.  A tech-savvy subset of users is going to organize code reviews and
publish their findings.  Non-technical users may then trust the users who
audited the source code.  This is no different from other free software
projects: nobody audits all the software that they use, ranging from the kernel
to the myriad of user space applications.

% One can hide bugs in plain sight.
The Underhanded C Coding Contest's~\cite{underhanded-c} goal was the
implementation of benign-looking code that was secretly malicious.  The contest
attracted numerous impressive submissions which showed that it is surprisingly
difficult to find bugs \emph{even if one knows} that there is a bug in a given
piece of code.  Analogously, the service provider could try to hide subtle, yet
critical bugs in the code to exfiltrate information from the enclave.  On top of
that, if the service provider ever gets caught, it may have plausible
deniability and pretend that the exfiltration bug was an honest programming
error.  We are unable to solve this class of attacks but we can mitigate it
by keeping the trusted computing base small.
