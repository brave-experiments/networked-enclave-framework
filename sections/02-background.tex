\section{Background}
\label{sec:background}

This section provides an overview of secure enclaves in general
(\S~\ref{sec:enclaves}) and AWS's implementation in particular
(\S~\ref{sec:nitro}).

\subsection{Secure Enclaves}
\label{sec:enclaves}

Computers operate on data that is at rest, in transit, and in use.  We have
well-understood and practical ways to protect data at rest (e.g., full disk
encryption) and in transit (e.g., TLS) but only limited solutions for data that
is in use.  Cryptography provides solutions in the form of fully homomorphic
encryption (FHE) and secure multiparty computation (MPC) but for many
applications, those building blocks remain too slow or cumbersome.  Trusted
execution environments---in particular in the form of ``secure
enclaves''---provide an alternative that is rooted in hardware and code.
Unlike FHE and MPC, enclaves perform at native (or near-native) execution speed
because they are general-purpose computing environments that are not limited to
the computation of carefully designed functions.  Conceptually, enclaves are
isolated execution environments that are shielded off from a computer's main
execution environment---typically by relying on hardware.  Enclaves offer
various security properties but in the context of this work, we rely on the
following three:

\begin{description}
  \item[Confidentiality] An unauthorized entity must not be able to see the
    data that an enclave is computing.

  \item[Integrity] An unauthorized entity must not be able to modify the data
    that the enclave is computing on, or the code it is running.

  \item[Attestability] A separate entity must be able to verify if the enclave
    is running a given piece of code.
\end{description}

Modern CPUs of major hardware vendors implement secure enclaves; Intel has SGX,
ARM has TrustZone, and AMD has SEV.  A frequent critique of these industry
efforts focuses on their proprietary nature. The community has a conceptual
understanding of the mechanisms behind enclaves but their exact hardware
implementation is not disclosed, which is in violation of Kerckhoffs's
principle and served as motivation towards an open source
enclave~\cite{Lee20a}.

In practice, enclaves promise to be useful in situations where a system must
process sensitive data while simultaneously be shielded off from the complexity
(and subsequent insecurity) of general-purpose computers. Financial and
health-related data are common examples of such situations.

\subsection{AWS Nitro Enclaves}
\label{sec:nitro}

In this work, we build on top of AWS's Nitro enclaves.  Nitro enclaves are
isolated and constrained virtual machines that run alongside an EC2 instance
that spawns and can communicate with the enclave.  Crucially, an enclave does
not share hardware resources with its parent EC2 image; it is guaranteed to
have its own CPU and memory which is physically isolated from the parent EC2
image.  The same hypervisor that isolates EC2 instances from each other also
isolates an EC2 instance from its Nitro enclave.  As far as computing resources
go, Nitro enclaves are essentially an independent computer, with its own
operating system, CPU, and memory, but \emph{without} a dedicated networking
interface.  All networking must go over the parent EC2 instance.  That is by
design, to constrain all communication with the enclave to a minimal VSOCK
interface.\footnote{Originally proposed for communication between a hypervisor
and its virtual machines, AWS repurposed the VSOCK interface to serve as
communication channel between an enclave and its parent EC2 image.}

% Workflow.
The development work flow begins with the creation of a Docker image that
contains the application that will run in the enclave.  Using Amazon's
nitro-cli command line tool, the developer then compiles the Docker image to an
enclave image file (EIF).  The compilation process results in a number of
\emph{measurement checksums} that uniquely identify the image itself, its
kernel, and application.  As we will discuss later in the paper, these
measurements are key to the remote attestation process.
%
Once the EIF is ready, the developer starts the enclave on an EC2 instance
using the nitro-cli command line tool.  The only way for the EC2 instance to
exchange data with the enclave is via the VSOCK interface.
