\section{Background}
\label{sec:background}

This section provides an overview of secure enclaves in general
(\S~\ref{sec:enclaves}) and AWS's implementation in particular
(\S~\ref{sec:nitro}).

\subsection{Secure Enclaves}
\label{sec:enclaves}

Computers operate on data that is at rest, in transit, and in use.  We have
well-understood and practical ways to protect data at rest (e.g., full disk
encryption) and in transit (e.g., TLS) but only limited solutions for data that
is in use.  Cryptography provides solutions in the form of fully homomorphic
encryption (FHE) and secure multiparty computation (MPC) but for many
applications, those building blocks remain too slow and cumbersome.  Trusted
execution environments---in particular in the form of ``secure
enclaves''---provide an alternative that is rooted in hardware and code.  Unlike
FHE and MPC, enclaves perform at native (or near-native) execution speed because
they are general-purpose computing environments that are not limited to the
computation of carefully designed functions.  Conceptually, enclaves are
isolated execution environments that are shielded off from a computer's main
execution environment, which runs the untrustworthy (from the enclave's point of
view) operating system.  Enclaves offer various security properties but in the
context of this work, we rely on the following three:

\begin{description}
  \item[Confidentiality] An unauthorized entity must not be able to observe the
    data that an enclave is computing.

  \item[Integrity] An unauthorized entity must not be able to modify the data
    that the enclave is computing on, or the code it is running.

  \item[Verifiability] A separate entity must be able to verify if the enclave
    is running the code that its operator claims it is running.
\end{description}

Modern CPUs of major hardware vendors implement secure enclaves: Intel has SGX,
ARM has TrustZone, and AMD has SEV.  A frequent critique of these industry
efforts focuses on their proprietary nature. The community has a conceptual
understanding of the mechanisms behind enclaves but their exact hardware
implementation is not disclosed, which served as motivation towards an open
source enclave~\cite{Lee20a}.  In practice, enclaves promise to be useful in
situations where a system must process sensitive data while simultaneously be
shielded off from the complexity (and subsequent insecurity) of general-purpose
computers.

\subsection{AWS Nitro Enclaves}
\label{sec:nitro}

In this work, we build on top of AWS's Nitro enclaves.  Nitro enclaves are
isolated and constrained virtual machines that run alongside an EC2 instance
that is responsible for starting the enclave, and communicating with it.
Crucially, an enclave does not share hardware resources with its parent EC2
image; it is guaranteed to have its own CPU and memory which is
isolated from the parent EC2 image by the same hypervisor that isolates EC2
instances from each other.
As far as computing resources go, Nitro enclaves are essentially an independent
computer, with its own operating system, CPU, and memory, but \emph{without} a
networking interface or persistent storage.  By design, the enclave's network
traffic must go through the parent EC2 instance, constrained to a minimal VSOCK
interface~\cite{vsock}. Originally proposed for communication between a
hypervisor and its virtual machines, AWS repurposed the VSOCK interface to serve
as communication channel between an enclave and its parent EC2 instance.  From a
developer's point of view, the VSOCK interface is a point-to-point interface
connecting the two.  On the address layer, 32-bit
context IDs take the role of IP addresses in VSOCK interfaces.  For example, the
enclave may have context ID 4 while its parent EC2 instance may have context ID
3.  On the transport layer, one can use the same protocols that one can use over
the IP-based address family; namely TCP, UDP, etc.

\begin{figure}[t]
  \centering
  \begin{tikzpicture}[node distance=20pt]

  \node [draw,
         align=center] (code) {Application\\source code};

  \node [draw,
         align=center,
         right=50pt of code] (docker) {Docker\\image};

  \node [draw,
         align=center,
         right=50pt of docker] (eif) {Enclave\\image};

  \draw[-latex] (code.east) -- (docker.west)
                node [midway, fill=white] {docker};
  \draw[-latex] (docker.east) -- (eif.west)
                node [midway, fill=white] {nitro-cli};

\end{tikzpicture}

  \caption{The development workflow for compiling enclave applications.
  Docker's command line tool compiles application source code into a Docker
  image, which is then compiled to an enclave image file using the nitro-cli
  command line tool.}
  \label{fig:dev-workflow}
\end{figure}

% Workflow.
Figure~\ref{fig:dev-workflow} illustrates the development workflow for enclave
applications: the workflow begins with the creation of a Docker image that
contains the application that will run in the enclave.  Using Amazon's
nitro-cli command-line tool, the developer then compiles the container image to
an enclave image file (EIF).  The compilation process results in a number of
\emph{measurement checksums} that uniquely identify the image itself, its
kernel, and application.  As we will discuss later in the paper, these
measurements are key to the remote attestation process.
%
Once the EIF is ready, the developer starts the enclave on an EC2 instance
using the nitro-cli command-line tool.  The only way for the EC2 instance to
exchange data with the enclave is via the VSOCK interface.
