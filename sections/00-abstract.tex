\begin{abstract}

In 2020, Amazon introduced Nitro enclaves: cloud-based secure enclaves that do
not share hardware with untrustworthy code, therefore promising resistance
against side channel attacks, which have plagued Intel's SGX for years.  While
their security properties are attractive, Nitro enclaves are difficult to
write code for and are not meant to be used as a networked service, which
greatly limits their potential.
%
  In this paper, we built \emph{\tool{}}---a tool kit that allows for convenient and
flexible use of Nitro enclaves by abstracting away complex aspects like remote
attestation and end-to-end encryption between an enclave and a remote client.
%
We demonstrate the practicality of \tool{} by designing and implementing three
novel systems that solve real-world problems: an enclave-enabled Tor bridge, an
enclave-enabled Web browser, and a system for ``infrastructure transparency'',
which securely makes public the configuration of, say, cloud providers.
%
Performance evaluations and our practical experience suggests that \tool{}
enables quick prototyping, is flexible enough to accommodate different use
cases, and inherits strong security and performance properties from the
underlying Nitro enclaves.

\end{abstract}
