\begin{abstract}

Enclave deployments often fail to simultaneously be
\emph{secure} (e.g., resistant to side channel attacks),
\emph{powerful} (i.e., as fast as an off-the-shelf server), and
\emph{flexible} (i.e., unconstrained by development hurdles).
%
In this paper, we present \tool{}, an open tool kit that enables the
development of enclave applications that satisfy all three properties.
%
We build \tool{} on top of the recently-proposed AWS Nitro Enclaves whose
architecture prevents side channel attacks by design, making \tool{} more
secure than comparable frameworks.  We abstract away the constrained
development model of Nitro Enclaves, making it possible to run unmodified
applications inside an enclave that have seamless and secure Internet
connectivity, all while making our code user-verifiable.
%
To demonstrate \tool{}'s flexibility, we design three enclave applications, each
a research contribution in its own right:
%
(i) we run a Tor bridge inside an enclave, making it resistant to protocol-level
deanonymization attacks;
%
(ii) we built a service for securely revealing infrastructure configuration,
empowering users to verify privacy promises like the discarding of IP addresses
at the edge;
%
(iii) and we move a Chromium browser into an enclave, thereby isolating its
attack surface from the user's system.
%
We find that \tool{} enables rapid prototyping and alleviates the deployment of
production-quality systems, paving the way toward usable and secure enclaves.

\end{abstract}
